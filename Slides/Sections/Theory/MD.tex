\subsection{Molecular Dynamics}
\begin{frame}{Molecular Dynamics}
Molecular Dynamics employs an empirical force field derived from classical physics in order to describe the system.
\begin{equation}
\label{eq:EMM}
\begin{split}
E_{Total} & = E_{Bonded} + E_{Non-Bonded} \\
E_{Bonded} & = E_{Bond} + E_{Angle} + E_{Dihedral} \\
E_{Non-Bonded} & = E_{Electrostatic} + E_{van ~der ~Waals} \\
\end{split}
\end{equation}
\begin{enumerate}
\item Potential energy of the system is calculated using \cref{eq:EMM}.
\end{enumerate}

\end{frame}

\begin{frame}{Molecular Dynamics}
Molecular Dynamics employs an empirical force field derived from classical physics in order to describe the system.
\begin{equation}
\label{eq:EMM1}
\begin{split}
F(x) = - \Delta E(x)
\end{split}
\end{equation}
\begin{enumerate}
\item Potential energy of the system is calculated using \cref{eq:EMM}.
\item Forces are derived by taking the negative gradient of the potential energy with respect to the atomic coordinates.

\end{enumerate}

\end{frame}

\begin{frame}{Molecular Dynamics}
Molecular Dynamics employs an empirical force field derived from classical physics in order to describe the system.
\begin{equation}
\label{eq:EMM2}
\begin{split}
F &= m a \\
x_{i+1} &= x_i +\delta_t v_i \\
\end{split}
\end{equation}
\begin{enumerate}
\item Potential energy of the system is calculated using \cref{eq:EMM}.
\item Forces are derived by taking the negative gradient of the potential energy with respect to the atomic coordinates.
\item Velocities are then calculated using Newton's laws of motion and the positions are updated with respect to the time step.
\end{enumerate}

\end{frame}