% ========
% Preamble
% ========


\documentclass[aspectratio=169]{beamer}
%\usetheme{_ross}
\usetheme{nottingham}

% Allows for gif implementation
\usepackage{animate}
\usepackage{xmpmulti}

% Afterpage flushing of floats
\usepackage{afterpage}

\setcounter{tocdepth}{2}

% -------------
% Maths support
% -------------
% Load maths packages
%% Typical packages
\usepackage{%
  amsmath,%
  amsthm,%
  amssymb,%
  mathrsfs,%
  mathtools,%
  gensymb,%
  braket,%
  mleftright,%
  xfrac,%
  interval,%
  rotating,%
  tabstackengine,%
}

% ------------------
% Scientific support
% ------------------
% Chemistry
\usepackage[version=4]{mhchem}

% Scientific typesetting
\usepackage{siunitx}
\DeclareSIUnit{\atomicunits}{a.u.}


% ----------
% Typography
% ----------
% Context-sensitive quotation marks
\usepackage[english=british]{csquotes}

% Micro-typography
\usepackage[%
  activate={true, nocompatibility},%
  final,%
  tracking=true,%
  factor=1100,%
]{microtype}

% Reduce spacing between sc characters
\SetTracking{encoding={*}, shape=sc}{40}

% Disable -- and --- ligatures for tt family
\DisableLigatures[-,>,']{encoding=*, family=tt*}

% --------
% Diagrams
% --------
% Graphics
\usepackage{graphicx}
\usepackage{transparent}


% TikZ & PGFPlots
\usepackage{pgfplots}

% Set up tikzexternalize
\usetikzlibrary{external}
\tikzexternalize[prefix=tikz/]

\tikzexternalize
\newif\iftikzex
\ifdefined\notikzex
  \tikzexfalse
\else
  \tikzextrue
\fi
%\tikzexternaldisable
\pgfplotsset{compat=1.16}

\usetikzlibrary{%
  calc,%
  luamath,%
  math,%
  positioning,%
  decorations.pathreplacing,%
  arrows.meta,%
  3d,%
  backgrounds,%
  patterns,%
  pgfplots.groupplots,%
}

\usepgfplotslibrary{%
  groupplots,%
  patchplots,%
  colormaps,%
}

\pgfplotscreateplotcyclelist{coloronly}{%
  {red},%
  {blue},%
  {black!60!green},%
  {black!20!orange},%
  {green!30!brown},%
  {blue!40!red},%
  {black!60!blue},%
  {black!40!yellow},%
  {red!50!pink},%
  {green!70!blue},%
}

% Set default plot linewidth
\tikzset{every picture/.style={line width=0.8pt}}

% Command for tikzexternalize/includegraphics
\makeatletter
\newcommand*{\useexternalfile}[4]{
  \iftikzex
    \tikzsetnextfilename{tikzoutput/#4-output}
    \scalebox{#1}{\input{\tikzexternal@filenameprefix#4.tikz.tex}}
  \else
    \includegraphics[scale=#1, trim=#2 0 #3 0]{\tikzexternal@filenameprefix tikzoutput/#4-output.pdf}
  \fi
}
\makeatother

%\tikzexternalize[optimize=false,prefix=tikz/]


% Plot conditionals
\usepackage{xifthen}

% Float positioning
\usepackage{placeins}

% ------------------------
% References & bookmarking
% ------------------------
% Footnotes
\usepackage[perpage]{footmisc}
\renewcommand{\thefootnote}{\alph{footnote}}

% Footnote kerning with punctuation
% Do not use with option multiple for footmisc since this package has a different way of handling multiple footnotes.
\usepackage{fnpct}

% Bibliography
\usepackage[%
  sorting=none,%
  style=science,%
  articletitle=true,%
  autopunct=true,%
  autocite=superscript,%
  doi=true,%
  dateabbrev=true,%
  url=true,%
  isbn=false,%
  backend=biber,%
]{biblatex}
%\addbibresource{references.bib}
\bibliography{references}

%% Allow breaks in DOI
\setcounter{biburlnumpenalty}{100}
\setcounter{biburllcpenalty}{9000}


%% biblatex: new cite command to combine \citeauthor and \autocite
\newcommand*{\authorcite}[1]{\citeauthor{#1}\autocite{#1}}

% Bookmarking
%% hyperref must be loaded last, but before glossaries.
%\usepackage[%
%  hyperfootnotes=false,%
%  hypertexnames=false,%
%  hidelinks,%
%]{hyperref}
\hypersetup{%
  bookmarksnumbered=true,%
  colorlinks=true,%
  citecolor=LinkColour,%
  linkcolor=LinkColour,%
}

% Cross-referencing
% cleveref must be loaded after hyperref
\usepackage[noabbrev, capitalise]{cleveref}


% ---------------
% List of symbols
% ---------------
% Load the glossaries package (after hyperref)
\usepackage[%
%  symbols,% create list of symbols
  abbreviations,% create list of abbreviations
  nomain,%
  nonumberlist,%
  nogroupskip,%
  nopostdot,%
%  automake%
]{glossaries-extra}


% Set acronym style
\setabbreviationstyle{long-short}

% Display glossaries in columns
\usepackage{glossary-mcols}

\setlength{\glsdescwidth}{\hsize}
\newglossarystyle{listofsymbolsstyle}{%
  \glossarystyle{list}%
  \renewcommand*{\glossaryentryfield}[5]{%
    \item[\glsentryitem{##1}\glstarget{##1}{##2}]%
    \hspace{0.5cm}##3\glspostdescription\space ##5}%
}

% Compute the widest entry just for the current glossary
\renewcommand{\glossarypreamble}{%
  \setlength{\parskip}{3pt}%
}
\glssetwidest[1]{symb:elecwfdet}

% Create an "ignored" list of symbols that will not be printed out
\newglossary[glignoredl]{ignored}{glignored}{glignoredin}{Ignored Glossary}

% Hyphenated long forms
\glsaddkey
  {hyphenated}        % new key
  {\relax}            % default value if "hyphenated" isn't used in \newglossaryentry
  {\glsentryhyphx}    % analogous to \glsentrytext
  {\Glsentryhyphx}    % analogous to \Glsentrytext
  {\glshyphx}         % analogous to \glstext
  {\Glshyphx}         % analogous to \Glstext
  {\GLShyphx}         % analogous to \GLStext
\newcommand{\GENglspostlinkhook}{%
  \ifglsused{\glslabel}{}{ (\glsentryshort{\glslabel})}\glsunset \glslabel}
\makeatletter
\newcommand\metadef[1]{%
  \expandafter\newcommand\csname gls#1\endcsname{%
    \@ifstar{\csname sgls#1\endcsname}{\csname ngls#1\endcsname}%
  }
  \@namedef{sgls#1}##1{{\let\glspostlinkhook \GENglspostlinkhook\expandafter\csname gls#1x\endcsname*{##1}}}%
  \@namedef{ngls#1}##1{{\let\glspostlinkhook \GENglspostlinkhook\expandafter\csname gls#1x\endcsname{##1}}}%
  \expandafter\newcommand\csname Gls#1\endcsname{%
    \@ifstar{\csname sGls#1\endcsname}{\csname nGls#1\endcsname}%
  }
  \@namedef{sGls#1}##1{{\let\glspostlinkhook \GENglspostlinkhook\expandafter\csname Gls#1x\endcsname*{##1}}}%
  \@namedef{nGls#1}##1{{\let\glspostlinkhook \GENglspostlinkhook\expandafter\csname Gls#1x\endcsname{##1}}}%
  \expandafter\newcommand\csname GLS#1\endcsname{%
    \@ifstar{\csname sGLS#1\endcsname}{\csname nGLS#1\endcsname}%
  }
  \@namedef{sGLS#1}##1{{\let\glspostlinkhook \GENglspostlinkhook\expandafter\csname GLS#1x\endcsname*{##1}}}%
  \@namedef{nGLS#1}##1{{\let\glspostlinkhook \GENglspostlinkhook\expandafter\csname GLS#1x\endcsname{##1}}}%
}
\makeatother

\metadef{hyph}


\makeglossaries
\loadglsentries{../symbols/acronyms}

% ------
% Tables
% ------
\usepackage{%
  booktabs,%
  multirow,%
  tabularx,%
  makecell,%
  pbox,%
  bigdelim,%
  adjustbox,%
  array,%
}



% ------------------------------
% Code listings and other frames
% ------------------------------
% Frames and boxes
%\usepackage[most]{tcolorbox}

\tcbuselibrary{%
  minted,%
  breakable
}

\tcbset{shield externalize}

% Non-copyable texts in PDF
\usepackage{%
  accsupp,%
}

\newcommand*{\noncopy}[1]{
  \BeginAccSupp{method=escape, ActualText={}}
  #1
  \EndAccSupp{}
}

% Listings environments

%% Minted style
\usemintedstyle{default}

%% Non-copyable line numbers
\renewcommand{\theFancyVerbLine}{\tiny\ttfamily\noncopy{\arabic{FancyVerbLine}}}

%% Non-copyable shell prompts
\newcommand{\LineWithPrompt}{%
  \def\FancyVerbFormatLine##1{\textcolor{red}{\noncopy{\$}}##1}%
}


%% Input file environment
\NewTCBListing[%
  auto counter,%
  number within=section,%
  Crefname={File}{Files},%
]{inpfile}{ O{} m m}{%
  breakable,%
  enhanced,%
  colback=PortlandStone,%
  colframe=NottinghamBlue,%
  listing only,%
%  before skip balanced=0pt,%
%  comment above* listing,%
%  comment={\centering \textbf{File~\thetcbcounter:} #2},%
  left=3pt,%
  right=3pt,%
  listing engine=minted,%
  minted language=text,%
  minted options={%
    breaklines,%
    autogobble,%
    stripall,%
    breaksymbolleft=\textcolor{gray}{\ensuremath{\noncopy{\hookrightarrow}}},%
    breaksymbolright=\textcolor{gray}{\ensuremath{\noncopy{\hookleftarrow}}},%
  },%
  title={\noncopy{\texttt{#3} file}},%
%  fit to height = 0.7\textheight ,%
  fit fontsize macros,%
  fontupper=\small,%
  #1
}

%% Shell command environment
\NewTCBListing[%
  auto counter,%
  number within=section,%
  Crefname={Listing}{Listings},%
]{bashcmd}{ s O{} m }{%
%  breakable,%
  enhanced,%
  colback=PortlandStone,%
  colframe=black,%
  listing only,%
  before skip balanced=0pt,%
  comment above* listing,%
  IfBooleanTF={#1}
    {}%
    {comment={\centering \textbf{Listing~\thetcbcounter:} #3\nopagebreak}},%
  left=3pt,%
  right=27pt,%
  listing engine=minted,%
  minted language=bash,%
  minted options={%
    breaklines,%
    autogobble,%
    stripall,%
    formatcom=\LineWithPrompt,%
    breaksymbolleft=\textcolor{gray}{\ensuremath{\noncopy{\hookrightarrow}}},%
    breaksymbolright=\textcolor{gray}{\ensuremath{\noncopy{\hookleftarrow}}},%
  },%
  title={\noncopy{\texttt{bash} command}},%
  #2
}


