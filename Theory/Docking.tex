\subsection{Docking}
    \paragraph{}
        Molecular docking is a method of estimating the interactions between two or more chemical species. There are many different types of docking available including covalent and non-covalent molecular docking. In this workshop, we are interested in the non-covalent docking between a protein and a ligand. Like molecular dynamics, docking uses a set of parameters which describe the molecules within a system and uses these to attempt to estimate how well a given orientation and position of a ligand will bind to a protein. This is known as the \enquote{scoring function} of the docking protocol. 
        
    \paragraph{}
Firstly the docking protocol samples translational, rotational and conformational space where the ligand is repositioned within the protein active site. This is often done using a Monte Carlo search in order to obtain a high number of different binding poses for the ligand.

    \paragraph{}
        The different poses are then scored with respect to their scoring function in order to estimate the binding affinity or energy of the interaction between the ligand and molecule. This scoring function can be parameterised to include a variety of chemical information including electronegativity, \gls{acr:vdw} interactions, hydrogen bonding and solvation effects. Each of the different poses generated by the initial sampling is then scored using this function and the best-scoring poses are saved for further evaluation. 

    \paragraph{}
        Higher-level docking programs can often take their best scoring poses and attempt to minimise them, often using a simple molecular dynamics forcefield. They then re-score the binding poses in order to get an improved estimate of the binding interactions.
    
    \paragraph{}
        Depending on what docking software you chose, the chemical information that it uses to score the binding affinity can change. It is therefore important to understand what parameters the docking score is based on in order to decide whether the docking software is suitable for your system. It is also known that docking accuracy can be heavily system dependent, so often a benchmark is required in order to assess the validity of any docking results. Docking can also be used to estimate the difference in binding energies between two different ligands, however often this is inaccurate and only suitable if the binding affinity difference is large.

        