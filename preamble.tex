% ========
% Preamble
% ========

% -----
% Class
% -----
\documentclass[%
  paper=a4,%
  11pt,%
  oneside,%
  DIV=10,%
  BCOR=0mm,%
  headinclude=true,%
  headings=optiontoheadtoc,%
  headsepline=true,%
  footinclude=false,%
  parskip=half,%
  listof=leveldown,%
  captions=tableheading,%
  toc=sectionentrywithdots,%
%  draft,%
]{scrartcl}


% -----------
% Page layout
% -----------
% Customise spacing between lines
\usepackage{setspace}
\onehalfspacing
\setlength{\parindent}{1.5em}

% Afterpage flushing of floats
\usepackage{afterpage}

% Headers and footers
\usepackage[automark]{scrlayer-scrpage}
\pagestyle{scrheadings}
\automark[subsection]{section}
\clearpairofpagestyles
\ihead{\headmark}
\ohead{\pagemark}
\ofoot[\pagemark]{}


% TOC
\usepackage{multicol}
\BeforeStartingTOC[toc]{\begin{multicols}{2}}
\AfterStartingTOC[toc]{\end{multicols}}
\setlength{\columnseprule}{.5pt}
\setlength\columnsep{30pt}
\setcounter{tocdepth}{2}

% Landscape
\usepackage{lscape}

% -------------
% Maths support
% -------------
% Load maths packages
%% Typical packages
\usepackage{%
  amsmath,%
  amsthm,%
  amssymb,%
  mathrsfs,%
  mathtools,%
  gensymb,%
  braket,%
  mleftright,%
  xfrac,%
  interval,%
  rotating,%
  tabstackengine,%
}

% Settings for packages
%% interval: change open fences to round brackets
\intervalconfig{soft open fences}

%% mleftright: redefine \left as \mleft and \right as \mright
\mleftright

%% amsmath: equation numbering by section
\numberwithin{equation}{section}

%% tabstackengine
\TABstackMath
\TABbinary


% ------------------
% Scientific support
% ------------------
% Chemistry
\usepackage[version=4]{mhchem}

% Scientific typesetting
\usepackage{siunitx}
\DeclareSIUnit{\atomicunits}{a.u.}

% Lists
\usepackage[inline]{enumitem}


% ----------
% Typography
% ----------
% Context-sensitive quotation marks
\usepackage[english=british]{csquotes}

% Micro-typography
\usepackage[%
  activate={true, nocompatibility},%
  final,%
  tracking=true,%
  factor=1100,%
]{microtype}

% Reduce spacing between sc characters
\SetTracking{encoding={*}, shape=sc}{40}

% Disable -- and --- ligatures for tt family
\DisableLigatures[-,>,']{encoding=*, family=tt*}

% -------------------
% Font configurations
% -------------------
% Font support
\usepackage[tuenc, no-math]{fontspec}
\defaultfontfeatures{%
%  Ligatures=TeX,%
%  Numbers=Lining,%
}

% Main font selection
\setmainfont{Alegreya}[%
  Renderer=OpenType,
  Ligatures=TeX
]
% Renderer=Basic is needed to disable -- ligatures for tt family
\setmonofont[Renderer=OpenType]{Ubuntu Mono}

% Maths font selection
\usepackage[%
  math-style=ISO,%
  bold-style=ISO,%
  warnings-off={mathtools-colon},%
]{unicode-math}
\setmathfont{STIX Two Math}

% Bold version
\setmathfont{STIX Two Math}[%
  version=bold,%
  FakeBold=3.3,%
]

% Stylistic set selection for script variant
\setmathfont{STIX Two Math}[%
  range={scr, bfscr},%
  StylisticSet=1,%
]

% New font family
\newfontfamily\alegreyalocal{Alegreya}[Renderer=OpenType]
\newfontfamily\alegreyasclocal{Alegreya SC}[Renderer=OpenType]

%% Scale maths script and scriptscript sizes for text size 8
%\DeclareMathSizes{8}{8}{5.5}{5}
%
%% Avoid font size substitution with xfrac
%\usepackage{anyfontsize}


% ----------------
% Language support
% ----------------
% polyglossia requires fontspec
\usepackage{polyglossia}
\setmainlanguage[variant=british]{english}

% Date and time format
\usepackage{datetime}
\newdateformat{monthyeardate}{%
  \monthname[\THEMONTH] \THEYEAR
}


% -------
% Colours
% -------
\usepackage[svgnames, x11names, luatex, rgb]{xcolor}

% Colour aliases
\colorlet{LinkColour}{DarkCyan!50!Black}
\colorlet{PartColour}{Blue4!70!Black}
\colorlet{SectionColour}{DodgerBlue4}
\colorlet{SubsectionColour}{DodgerBlue3}
\colorlet{SubsubsectionColour}{DodgerBlue2}
\colorlet{ParagraphColour}{DodgerBlue1!70!Black}
\colorlet{HeaderFooterColour}{DeepSkyBlue4!70!Black}


% --------
% Diagrams
% --------
% Graphics
\usepackage{graphicx}

% TikZ & PGFPlots
\usepackage{pgfplots}

% Set up tikzexternalize
\usetikzlibrary{external}
\tikzexternalize
\newif\iftikzex
\ifdefined\notikzex
  \tikzexfalse
\else
  \tikzextrue
\fi
\tikzexternaldisable

\pgfplotsset{compat=1.17}

\usetikzlibrary{%
  calc,%
  luamath,%
  math,%
  positioning,%
  decorations.pathreplacing,%
  arrows.meta,%
  3d,%
  backgrounds,%
  patterns,%
}

\usepgfplotslibrary{%
  groupplots,%
  patchplots,%
  colormaps,%
}

\pgfplotscreateplotcyclelist{coloronly}{%
  {red},%
  {blue},%
  {black!60!green},%
  {black!20!orange},%
  {green!30!brown},%
  {blue!40!red},%
  {black!60!blue},%
  {black!40!yellow},%
  {red!50!pink},%
  {green!70!blue},%
}

% Command for tikzexternalize/includegraphics
\makeatletter
\newcommand*{\useexternalfile}[4]{
  \iftikzex
    \tikzsetnextfilename{tikzoutput/#4-output}
    \scalebox{#1}{\input{\tikzexternal@filenameprefix#4.tikz.tex}}
  \else
    \includegraphics[scale=#1, trim=#2 0 #3 0]{\tikzexternal@filenameprefix tikzoutput/#4-output.pdf}
  \fi
}
\makeatother

% Plot conditionals
\usepackage{xifthen}

% Float positioning
\usepackage{placeins}

% ------
% Tables
% ------
\usepackage{%
  booktabs,%
  multirow,%
  tabularx,%
  makecell,%
  pbox,%
  bigdelim,%
  array,%
}
\usepackage[export]{adjustbox}

% --------
% Captions
% --------
% Bold pre-texts in captions
\usepackage[%
  format=plain,%
  font={small, stretch=1.1},%
  labelfont=bf,%
  skip=5pt,%
]{caption}

% Subcaption support
\usepackage[%
  labelfont=rm,%
  subrefformat=parens,%
]{subcaption}

\captionsetup[subfigure]{justification=centering}
\captionsetup[subtable]{justification=centering}


% ------------------------
% References & bookmarking
% ------------------------
% Footnotes
\usepackage[perpage]{footmisc}
\renewcommand{\thefootnote}{\alph{footnote}}

% Footnote kerning with punctuation
% Do not use with option multiple for footmisc since this package has a different way of handling multiple footnotes.
\usepackage{fnpct}

% Bibliography
\usepackage[%
  sorting=none,%
  style=science,%
  articletitle=true,%
  autopunct=true,%
  autocite=superscript,%
  doi=true,%
  dateabbrev=true,%
  url=true,%
  isbn=false,%
  backend=biber,%
]{biblatex}
%\addbibresource{bib/hx_magnetic_symmetry.bib}
\bibliography{references}

%% Allow breaks in DOI
\setcounter{biburlnumpenalty}{100}
\setcounter{biburllcpenalty}{9000}


%% biblatex: new cite command to combine \citeauthor and \autocite
\newcommand*{\authorcite}[1]{\citeauthor{#1}\autocite{#1}}

% Bookmarking
%% hyperref must be loaded last, but before glossaries.
\usepackage[%
  hyperfootnotes=false,%
  hypertexnames=false,%
  hidelinks,%
]{hyperref}
\hypersetup{%
  bookmarksnumbered=true,%
  colorlinks=true,%
  citecolor=LinkColour,%
  linkcolor=LinkColour,%
}

% Cross-referencing
% cleveref must be loaded after hyperref
\usepackage[noabbrev, capitalise]{cleveref}

% ---------------
% List of symbols
% ---------------
% Load the glossaries package (after hyperref)
\usepackage[%
  symbols,% create list of symbols
  abbreviations,% create list of abbreviations
  nomain,%
  nonumberlist,%
  nogroupskip,%
  nopostdot,%
  automake%
]{glossaries-extra}

% Set acronym style
\setabbreviationstyle{long-short}

% Display glossaries in columns
\usepackage{glossary-mcols}

\setlength{\glsdescwidth}{\hsize}
\newglossarystyle{listofsymbolsstyle}{%
  \glossarystyle{list}%
  \renewcommand*{\glossaryentryfield}[5]{%
    \item[\glsentryitem{##1}\glstarget{##1}{##2}]%
    \hspace{0.5cm}##3\glspostdescription\space ##5}%
}

% Compute the widest entry just for the current glossary
\renewcommand{\glossarypreamble}{%
  \setlength{\parskip}{3pt}%
}
\glssetwidest[1]{symb:elecwfdet}

% Create an "ignored" list of symbols that will not be printed out
\newglossary[glignoredl]{ignored}{glignored}{glignoredin}{Ignored Glossary}

% Hyphenated long forms
\glsaddkey
  {hyphenated}        % new key
  {\relax}            % default value if "hyphenated" isn't used in \newglossaryentry
  {\glsentryhyphx}    % analogous to \glsentrytext
  {\Glsentryhyphx}    % analogous to \Glsentrytext
  {\glshyphx}         % analogous to \glstext
  {\Glshyphx}         % analogous to \Glstext
  {\GLShyphx}         % analogous to \GLStext
\newcommand{\GENglspostlinkhook}{%
  \ifglsused{\glslabel}{}{ (\glsentryshort{\glslabel})}\glsunset \glslabel}
\makeatletter
\newcommand\metadef[1]{%
  \expandafter\newcommand\csname gls#1\endcsname{%
    \@ifstar{\csname sgls#1\endcsname}{\csname ngls#1\endcsname}%
  }
  \@namedef{sgls#1}##1{{\let\glspostlinkhook \GENglspostlinkhook\expandafter\csname gls#1x\endcsname*{##1}}}%
  \@namedef{ngls#1}##1{{\let\glspostlinkhook \GENglspostlinkhook\expandafter\csname gls#1x\endcsname{##1}}}%
  \expandafter\newcommand\csname Gls#1\endcsname{%
    \@ifstar{\csname sGls#1\endcsname}{\csname nGls#1\endcsname}%
  }
  \@namedef{sGls#1}##1{{\let\glspostlinkhook \GENglspostlinkhook\expandafter\csname Gls#1x\endcsname*{##1}}}%
  \@namedef{nGls#1}##1{{\let\glspostlinkhook \GENglspostlinkhook\expandafter\csname Gls#1x\endcsname{##1}}}%
  \expandafter\newcommand\csname GLS#1\endcsname{%
    \@ifstar{\csname sGLS#1\endcsname}{\csname nGLS#1\endcsname}%
  }
  \@namedef{sGLS#1}##1{{\let\glspostlinkhook \GENglspostlinkhook\expandafter\csname GLS#1x\endcsname*{##1}}}%
  \@namedef{nGLS#1}##1{{\let\glspostlinkhook \GENglspostlinkhook\expandafter\csname GLS#1x\endcsname{##1}}}%
}
\makeatother

\metadef{hyph}


\makeglossaries
%\loadglsentries{symbols/symbols}
\loadglsentries{symbols/acronyms}


% ---------------
% Design features
% ---------------
% Counter depth
\setcounter{secnumdepth}{\subsubsectionnumdepth}
% Heading styles
%% Remove all end-of-counter dots
\renewcommand*{\autodot}{}

%% Use alphabets for subsubsection
\renewcommand\thesubsection{\thesection.\alph{subsection}}
\renewcommand\thesubsubsection{\thesubsection.\roman{subsubsection}}

%% KOMA-Script markups: adjust fonts and colours
\addtokomafont{disposition}{\alegreyalocal}
\addtokomafont{part}{\alegreyasclocal\Huge\color{PartColour}}
\addtokomafont{partnumber}{\alegreyasclocal\color{PartColour}}
\addtokomafont{section}{\Large\color{SectionColour}}
\addtokomafont{subsection}{\color{SubsectionColour}}
\addtokomafont{subsubsection}{\color{SubsubsectionColour}}
\addtokomafont{paragraph}{\itshape\color{ParagraphColour}}
\addtokomafont{subparagraph}{\bfseries}

%% KOMA_Script headers and footers: adjust fonts and colours
\addtokomafont{pageheadfoot}{\alegreyalocal\color{HeaderFooterColour}}
\addtokomafont{headsepline}{\color{HeaderFooterColour}}
\addtokomafont{pagenumber}{\alegreyalocal\color{HeaderFooterColour}}



%% Put section numbers in the margin
\renewcommand*{\sectionformat}{%
  \llap{\thesection\autodot\enskip}%
}

%% Put subsection numbers in the margin
\renewcommand*{\subsectionformat}{%
  \llap{\thesubsection\autodot\enskip}%
}

%% Put subsubsection numbers in the margin
\renewcommand*{\subsubsectionformat}{%
  \llap{\thesubsubsection\autodot\enskip}%
}


%% Adjust spacing around headings
\RedeclareSectionCommand[%
  afterskip=.25\baselineskip,%
]{section}
\RedeclareSectionCommand[%
  beforeskip=-.1\baselineskip,%
  afterskip=.1\baselineskip,%
]{subsection}
\RedeclareSectionCommand[%
  beforeskip=-.1\baselineskip,%
  afterskip=.1\baselineskip,%
]{subsubsection}
\RedeclareSectionCommand[%
  beforeskip=-.1\baselineskip,%
  afterskip=.1\baselineskip,%
]{paragraph}


% ---------------
% Special symbols
% ---------------
% Operators
\DeclareMathOperator{\im}{im}
\DeclareMathOperator{\id}{id}
\DeclareMathOperator{\tr}{tr}
\DeclareMathOperator{\spn}{span}
\DeclareMathOperator{\Ln}{Ln}
\DeclareMathOperator{\diag}{diag}
\DeclareMathOperator{\Sym}{Sym}
\DeclareMathOperator{\Arg}{Arg}

% Commands
%% Differential operator
\newcommand*{\D}{\symup{d}}

%% Transpose
\newcommand*{\T}{\symsfup{T}}

%% Vector operator
\newcommand*{\vecop}[1]{\hat{\symbfit{#1}}}

%% Latin abbreviations
\usepackage{xspace}
\newcommand*{\eg}{\textit{e.g.}\@\xspace}
\newcommand*{\ie}{\textit{i.e.}\@\xspace}
\newcommand*{\ca}{\textit{ca.}\@\xspace}
\newcommand*{\cf}{\textit{cf.}\@\xspace}

\makeatletter
\newcommand*{\etc}{%
    \@ifnextchar{.}%
        {\textit{etc}}%
        {\textit{etc}.\@\xspace}%
}
\makeatother



% ------------------------------
% Code listings and other frames
% ------------------------------
% Frames and boxes
\usepackage[most]{tcolorbox}

\tcbuselibrary{%
  minted,%
  breakable
}

%\usepackage{upquote}

% Non-copyable texts in PDF
\usepackage{%
  accsupp,%
}

\newcommand*{\noncopy}[1]{
  \BeginAccSupp{method=escape, ActualText={}}
  #1
  \EndAccSupp{}
}

% Listings environments

%% Minted style
\usemintedstyle{default}

%% Non-copyable line numbers
\renewcommand{\theFancyVerbLine}{\tiny\ttfamily\noncopy{\arabic{FancyVerbLine}}}

%% Non-copyable shell prompts
\newcommand{\LineWithPrompt}{%
  \def\FancyVerbFormatLine##1{\textcolor{red}{\noncopy{\$}}##1}%
}

%% Shell command environment
\NewTCBListing[%
  auto counter,%
  number within=section,%
  Crefname={Listing}{Listings},%
]{bashcmd}{ s O{} m }{%
%  breakable,%
  enhanced,%
  colback=Seashell,%
  colframe=Grey!30!black,%
  listing only,%
  before skip balanced=0pt,%
  comment above* listing,%
  IfBooleanTF={#1}
    {}%
    {comment={\centering \textbf{Listing~\thetcbcounter:} #3\nopagebreak}},%
  left=3pt,%
  right=27pt,%
  listing engine=minted,%
  minted language=bash,%
  minted options={%
    breaklines,%
    autogobble,%
    stripall,%
    formatcom=\LineWithPrompt,%
    breaksymbolleft=\textcolor{gray}{\ensuremath{\noncopy{\hookrightarrow}}},%
    breaksymbolright=\textcolor{gray}{\ensuremath{\noncopy{\hookleftarrow}}},%
  },%
  title={\noncopy{\texttt{bash} command}},%
  #2
}

%% Shell output environment
\NewTCBListing[%
  use counter from=bashcmd,%
  Crefname={Listing}{Listings},%
]{bashoutput}{ s O{} m }{%
  enhanced,%
  colback=PapayaWhip,%
  colframe=Grey!30!black,%
  listing only,%
  before skip balanced=0pt,%
  comment above* listing,%
  IfBooleanTF={#1}
    {}%
    {comment={\centering \textbf{Listing~\thetcbcounter:} #3\nopagebreak}},%
  left=3pt,%
  right=3pt,%
  listing engine=minted,%
  minted language=bash,%
  minted options={%
    breaklines,%
    autogobble,%
    stripall,%
    breaksymbolleft=\textcolor{gray}{\ensuremath{\noncopy{\hookrightarrow}}},%
    breaksymbolright=\textcolor{gray}{\ensuremath{\noncopy{\hookleftarrow}}},%
  },%
  title={\noncopy{\texttt{bash} output}},%
  #2
}

%% Slurm file environment
% \readslurmscript[<options>]{caption}{filepath}
\NewTCBInputListing[%
  use counter from=bashcmd,%
  Crefname={Listing}{Listings},%
]{\readslurmscript}{ O{} m v }{%
%  breakable,%
  enhanced,%
  colback=Cornsilk,%
  colframe=Grey!30!black,%
  listing only,%
  before skip balanced=0pt,%
  comment above* listing,%
  comment={\centering \textbf{Listing~\thetcbcounter:} #2\nopagebreak},%
  left=3pt,%
  right=3pt,%
  listing engine=minted,%
  minted language=slurm,%
  minted options={%
    breaklines,%
    autogobble,%
    stripall,%
    linenos,%
    numbersep=3mm,%
    breaksymbolleft=\textcolor{gray}{\ensuremath{\noncopy{\hookrightarrow}}},%
    breaksymbolright=\textcolor{gray}{\ensuremath{\noncopy{\hookleftarrow}}},%
  },
  listing file={#3},%
  title={\noncopy{\texttt{SLURM} submission script:}\texttt{#3}},%
  #1
}

%% Shell script environment
% \readbashscript[<options>]{caption}{filepath}
\NewTCBInputListing[%
  use counter from=bashcmd,%
  Crefname={Listing}{Listings},%
]{\readbashscript}{ O{} m m }{%
  breakable,%
  enhanced,%
  colback=Cornsilk,%
  colframe=Grey!30!black,%
  listing only,%
  before skip balanced=0pt,%
  comment above* listing,%
  comment={\centering \textbf{Listing~\thetcbcounter:} #2\nopagebreak},%
  left=3pt,%
  right=3pt,%
  listing engine=minted,%
  minted language=bash,%
  minted options={%
    breaklines,%
    autogobble,%
    stripall,%
    linenos,%
    numbersep=3mm,%
    breaksymbolleft=\textcolor{gray}{\ensuremath{\noncopy{\hookrightarrow}}},%
    breaksymbolright=\textcolor{gray}{\ensuremath{\noncopy{\hookleftarrow}}},%
  },
  listing file={#3},%
  title={\noncopy{\texttt{bash} script:}\texttt{#3}},%
  #1
}

%% Generic text file environment
% \readbashscript[<options>]{file type}{caption}{filepath}
\NewTCBInputListing[%
  use counter from=bashcmd,%
  Crefname={Listing}{Listings},%
]{\readtextfile}{ O{} m m m }{%
  breakable,%
  enhanced,%
  colback=Cornsilk,%
  colframe=Grey!30!black,%
  listing only,%
  before skip balanced=0pt,%
  comment above* listing,%
  comment={\centering \textbf{Listing~\thetcbcounter:} #3\nopagebreak},%
  left=3pt,%
  right=3pt,%
  listing engine=minted,%
  minted language=text,%
  minted options={%
    breaklines,%
    autogobble,%
    stripall,%
    linenos,%
    numbersep=3mm,%
    breaksymbolleft=\textcolor{gray}{\ensuremath{\noncopy{\hookrightarrow}}},%
    breaksymbolright=\textcolor{gray}{\ensuremath{\noncopy{\hookleftarrow}}},%
  },
  listing file={#4},%
  title={\noncopy{#2:}\texttt{#4}},%
  #1
}

%% Task environment
\DeclareTColorBox[%
  auto counter,%
  number within=section,%
  Crefname={Task}{Tasks},%
]{task}{ O{} m }{%
  enhanced,%
%  breakable,%
  title={Task~\thetcbcounter: #2\nopagebreak},%
  colframe=Grey!10!black,%
  colback=LightGoldenrod1,%
  colbacktitle=DarkGoldenrod1,%
  fonttitle=\bfseries,%
  coltitle=black,%
  left=3pt,%
  right=6pt,%
  top=12pt,%
  attach boxed title to top left={%
    xshift=9mm,%
    yshift=-0.25mm-\tcboxedtitleheight/2,%
    yshifttext=2mm-\tcboxedtitleheight/2,%
  },%
  boxed title style={%
    boxrule=0.5mm,
    frame code={%
      \path[tcb fill frame] ([xshift=-4mm]frame.west)
-- (frame.north west) -- (frame.north east) -- ([xshift=4mm]frame.east)
-- (frame.south east) -- (frame.south west) -- cycle;%
    },%
    interior code={%
      \path[tcb fill interior] ([xshift=-2mm]interior.west)
-- (interior.north west) -- (interior.north east)
-- ([xshift=2mm]interior.east) -- (interior.south east) -- (interior.south west)
-- cycle;%
    }%
  },%
  #1
}

%% Input file environment
\NewTCBListing[%
  auto counter,%
  number within=section,%
  Crefname={File}{Files},%
]{inpfile}{ O{} m m}{%
  breakable,%
  enhanced,%
  colback=Seashell,%
  colframe=Grey!30!black,%
  listing only,%
%  before skip balanced=0pt,%
  comment above* listing,%
  comment={\centering \textbf{File~\thetcbcounter:} #2},%
  left=3pt,%
  right=3pt,%
  listing engine=minted,%
  minted language=text,%
  minted options={%
    breaklines,%
    autogobble,%
    stripall,%
    breaksymbolleft=\textcolor{gray}{\ensuremath{\noncopy{\hookrightarrow}}},%
    breaksymbolright=\textcolor{gray}{\ensuremath{\noncopy{\hookleftarrow}}},%
  },%
  title={\noncopy{\texttt{#3} file}},%
  #1
}

% -----------
% KOMA-Script
% -----------
% scrhack to prevent KOMA's "\float@addtolists detected!"
\usepackage{scrhack}
% Recalculate page margin based on loaded font
\recalctypearea